\documentclass[10pt,a4paper]{article}
\usepackage[utf8]{inputenc}
\usepackage{amsmath}
\usepackage{amsfonts}
\usepackage{amssymb}
\author{Joel David Hamkins}
\title{MASTER CLASS TUTORIAL ON POTENTIALISM
}

\date{}

\begin{document}
\maketitle
\abstract{
I shall give a master class tutorial on potentialism, an introduction to the general theory of potentialism that has been emerging in recent work, often developed as a part of research on set-theoretic pluralism, but just as often branching out to a broader application. Although the debate between potentialism and actualism in the philosophy of mathematics goes back to Aristotle, recent work divorces the potentialist idea from its connection with infinity and undertakes a more general analysis of possible mathematical universes of any kind. Any collection of mathematical structures forms a potentialist system when equipped with an accessibility relation (refining the submodel relation), and one can define the modal operators of possibility $\Diamond \varphi$, true at a world when $\varphi$ is true in some larger world, and necessity $\Box \varphi$, true in a world when $\varphi$ is true in all larger worlds. The project is to understand the structures more deeply by understanding their modal nature in the context of a potentialist system. The rise of modal model theory investigates very general instances of potentialist system, for sets, graphs, fields, and so on. Potentialism for the models of arithmetic often connects with deeply philosophical ideas on ultrafinitism. And the spectrum of potentialist systems for the models of set theory reveals fundamentally different conceptions of set-theoretic pluralism and possibility.
}
\end{document}